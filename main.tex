\documentclass[a4paper]{article}

\usepackage[english]{babel}
\usepackage[utf8]{inputenc}
\usepackage{amsmath}
\usepackage{graphicx}
\usepackage[colorinlistoftodos]{todonotes}
\usepackage{hyperref}
\usepackage{listings}
\usepackage[numbers]{natbib}

\usepackage{booktabs} % To thicken table lines

\title{Train a Smartcab How to Drive}

\author{Uirá Caiado}

\date{\today}

\begin{document}

\maketitle

\begin{abstract}
A smartcab is a self-driving car from the not-so-distant future that ferries people from one arbitrary location to another. In this project, I will design the AI driving agent for the smartcab using reinforcement learning. This area of machine learning\footnote{Source: \url{https://en.wikipedia.org/wiki/Reinforcement_learning}} is inspired by behaviorist psychology and consists in training the agent by reward and punishment without needing to specify how the task is to be achieved. The agent should learn an optimal policy for driving on city roads, obeying traffic rules correctly, and trying to reach the destination within a goal time.
\end{abstract}

%%%%%%%%%%%%%%%%%%%%%%%%%%%%%%%%%%%%%%%%%%%%%%%%%%%%%%%%%%%%%%%%%%%%%%%%%%%%%%%%%%%%%%%%
%% INTRODUCTION
%%%%%%%%%%%%%%%%%%%%%%%%%%%%%%%%%%%%%%%%%%%%%%%%%%%%%%%%%%%%%%%%%%%%%%%%%%%%%%%%%%%%%%%%

\section{Introduction}
\label{sec:introduction}
In this section, I will present a brief introduction to reinforcement learning and to the goal of the project.

\subsection{Reinforcement Learning}
As explained by \cite{Mohri_2012}, reinforcement learning is the study of planning and learning in a scenario where a learner (or agent) actively interacts with the environment to achieve a particular goal. The achievement of the agent's goal is typically measured by the reward he receives from the environment and which he seeks to maximize.

\cite{Kaelbling_1996} state that the most significant difference between reinforcement learning and supervised learning is that there is no presentation of input/output pairs. Instead, they explained that after choosing an action, the agent is told the immediate reward and the following state, but is not told which action would have been in its best long-term interests. It is necessary for the agent to gather useful experience about the possible system states, actions, transitions and rewards actively to act optimally.

Defining a general formulation of the problem based on a Markov Decision Process (MDP), as proposed by \cite{Mitchell}, the agent can perceive a set $S$ os distinct states of its environment and has a set $A$ of actions that it can perform. So, at each discrete time step $t$, the agent senses the current state $s_t$ and choose to take an action $a_t$. The environment responds by giving the agent a reward $r_t=r(s_t, a_t)$ and by producing the succeeding state $s_{t+1}=\delta(s_t, a_t)$. The functions $r$ and $\delta$ only depend on the current state and action (it is memoryless\footnote{Source: \url{https://en.wikipedia.org/wiki/Markov_process}}), are part of the environment and are not necessarily known to the agent.

The task of the agent is to learn a policy $\pi$ that maps each state to an action ($\pi: S \rightarrow A$), selecting its next action $a_t$ based solely on the current observed state $s_t$, that is $\pi(s_t)=a_t$. The optimal policy, or control strategy, is the one that produces the greatest possible cumulative reward over time. So, stating that:

$$V^{\pi}(s_t)= r_t + \gamma r_{t+1} + \gamma^2 r_{t+1} + ... = \sum_{i=0}^{\infty} \gamma^{i} r_{t+i}$$

Where $V^{\pi}(s_t)$ is also called the discounted cumulative reward and it represents the cumulative value achieved by following an policy $\pi$ from an initial state $s_t$ and $\gamma \in [0, 1]$ is a constant that determines the relative value of delayed versus immediate rewards. If we set $\gamma=0$, only immediate rewards is considered. As $\gamma \rightarrow 1$, future rewards are given greater emphasis relative to immediate reward. The optimal policy $\pi^{*}$ that will maximizes $V^{\pi}(s_t)$ for all states $s$ can be written as:

$$\pi^{*} = \underset{\pi}{\arg \max} \, V^{\pi} (s)\,\,\,\,\,, \,\, \forall s$$

As learning $\pi^{*}: S \rightarrow A$ directly is difficult because the available training data does not provide training examples of the form $(s, a)$, in the next sections I will implement the Q-learning algorithm for estimating the optimal policy.

\subsection{The Goal}

The goal of this project\footnote{Source: \url{https://goo.gl/BZdyLo}} is to design the AI driving agent for the smartcab, that operates in an idealized grid-like city, with roads going North-South and East-West. Other vehicles may be present on the streets, but no pedestrians. There is a traffic light at each intersection that can be in one of two states: North-South open or East-West open. US right-of-way rules apply: On a green light, you can turn left only if there is no oncoming traffic at the intersection coming straight. On a red light, you can turn right if there is no oncoming traffic turning left or traffic from the left going straight.

We are told to assume that a higher-level planner assigns a route to the smartcab, splitting it into waypoints at each intersection. The time in this world is quantized. At any instant, the smartcab is at some intersection. Therefore, the next waypoint is always either one block straight ahead, one block left, one block right, one block back or exactly there (reached the destination).

The smartcab is able to sense whether the traffic light is green for its direction of movement and whether there is a car at the intersection on each of the incoming roadways (and which direction they are trying to go). In addition to this, each trip has an associated timer that counts down every time step. If the timer is at 0 and the destination has not been reached, the trip is over, and a new one may start.

It should receive the inputs mentioned above at each time step t, and generate an output move, that consists on to stay put at the current intersection, move one block forward, one block left, or one block right (no backward movement).

The smartcab also should receive a reward for each successfully completed trip. A trip is considered “successfully completed” if the passenger is dropped off at the desired destination within a pre-specified time bound. It also gets a smaller reward for each correct move executed at an intersection. It gets a minor penalty for a wrong move and a larger penalty for violating traffic rules and/or causing an accident.

Based on the rewards and penalties it gets, the agent should learn an optimal policy for driving on city roads, obeying traffic rules correctly, and trying to reach the destination within a goal time.

%%%%%%%%%%%%%%%%%%%%%%%%%%%%%%%%%%%%%%%%%%%%%%%%%%%%%%%%%%%%%%%%%%%%%%%%%%%%%%%%%%%%%%%%
%% IMPLEMENT A BASIC DRIVING AGENT
%%%%%%%%%%%%%%%%%%%%%%%%%%%%%%%%%%%%%%%%%%%%%%%%%%%%%%%%%%%%%%%%%%%%%%%%%%%%%%%%%%%%%%%%

\section{Implement a Basic Driving Agent}
\label{sec:implement_driving_agent}
In this section, ...


%%%%%%%%%%%%%%%%%%%%%%%%%%%%%%%%%%%%%%%%%%%%%%%%%%%%%%%%%%%%%%%%%%%%%%%%%%%%%%%%%%%%%%%%
%% IDENTIFY AND UPDATE STATE
%%%%%%%%%%%%%%%%%%%%%%%%%%%%%%%%%%%%%%%%%%%%%%%%%%%%%%%%%%%%%%%%%%%%%%%%%%%%%%%%%%%%%%%%

\section{Identify and Update State}
\label{sec:identify_update_state}
In this section, ...


%%%%%%%%%%%%%%%%%%%%%%%%%%%%%%%%%%%%%%%%%%%%%%%%%%%%%%%%%%%%%%%%%%%%%%%%%%%%%%%%%%%%%%%%
%% IMPLEMENT Q-LEARNING
%%%%%%%%%%%%%%%%%%%%%%%%%%%%%%%%%%%%%%%%%%%%%%%%%%%%%%%%%%%%%%%%%%%%%%%%%%%%%%%%%%%%%%%%

\section{Implement Q-Learning}
\label{sec:implement_q_learning}
In this section, ...


%%%%%%%%%%%%%%%%%%%%%%%%%%%%%%%%%%%%%%%%%%%%%%%%%%%%%%%%%%%%%%%%%%%%%%%%%%%%%%%%%%%%%%%%
%% ENHANCE DRIVING AGENT
%%%%%%%%%%%%%%%%%%%%%%%%%%%%%%%%%%%%%%%%%%%%%%%%%%%%%%%%%%%%%%%%%%%%%%%%%%%%%%%%%%%%%%%%

\section{Enhance the Driving Agent}
\label{sec:enhance_driving_agent}
In this section, ...


%%%%%%%%%%%%%%%%%%%%%%%%%%%%%%%%%%%%%%%%%%%%%%%%%%%%%%%%%%%%%%%%%%%%%%%%%%%%%%%%%%%%%%%%
%% CONCLUSION
%%%%%%%%%%%%%%%%%%%%%%%%%%%%%%%%%%%%%%%%%%%%%%%%%%%%%%%%%%%%%%%%%%%%%%%%%%%%%%%%%%%%%%%%

\section{Conclusion}
\label{sec:conclusion}

As stated in ...

%%%%%%%%%%%%%%%%%%%%%%%%%%%%%%%%%%%%%%%%%%%%%%%%%%%%%%%%%%%%%%%%%%%%%%%%%%%%%%%%%%%%%%%%
%% REFLECTION
%%%%%%%%%%%%%%%%%%%%%%%%%%%%%%%%%%%%%%%%%%%%%%%%%%%%%%%%%%%%%%%%%%%%%%%%%%%%%%%%%%%%%%%%

\section{Reflection}
\label{sec:reflection}
Given that ...




\bibliographystyle{plain}
% or try abbrvnat or unsrtnat
\bibliography{bibliography/biblio.bib}
\end{document}
